\documentclass[openany,a4paper,12pt]{article}

\usepackage{amssymb,tikz}
\usepackage{amsfonts}
\usepackage{amsmath}
\usepackage{amsthm}
\usepackage{mathtools}
\usepackage{xcolor}
\usepackage{listings}
\usepackage{xepersian}

\settextfont[Scale=1]{XB Zar}

\lstset{basicstyle=\ttfamily\small,
  showstringspaces=false,
  commentstyle=\color{red},
  keywordstyle=\color{blue}
}

\renewcommand{\baselinestretch}{1.4}
\textwidth=15.5cm
\textheight=25cm
\voffset=-3cm
\oddsidemargin=.5cm
\evensidemargin=.5cm
\setlength{\unitlength}{10mm}
\renewcommand{\baselinestretch}{1.5} 

\newcommand{\rom}[1]{{\romannumeral #1}}

\pagestyle{empty}

\begin{document}
\begin{center}

\textbf{گزارش کار طراحی چندریسه‌ای}

استاد درس: دکتر مهدی کارگهی

پاییز ۹۹
\end{center}

در این پروژه به حل یک مسئلهٔ سادهٔ یادگیری ماشین می‌پردازیم. در ابتدا این کار را به صورت سری انجام می‌دهیم و در فاز دوم، سعی می‌کنم به کمک کتابخانهٔ pthread از قابلیت پردازش موازی سیستم استفاده کنیم و زمان صرف‌شده برای اجرای برنامه را کاهش دهیم.
برای موازی‌سازی، فایل trains.csv را به ۴ فایل با اندازه یکسان تقسیم کرده و هر یک را توسط یک پردازه مستقل می‌خوانیم و در یک بردار ذخیر می‌کنیم. در اینتها این بردارها ره به هم متصل کرده و باقی مراحل را مانند روش سری انجام می‌دهیم.
اسکریپت زیر برای اجرای برنامه‌های سری و موازی روی داده یکسان و مقایسه زمان اجرای هر یک نوشته شده.

\LTR
\begin{lstlisting}[language=bash]
#!/bin/bash
echo "Making serial"
(cd serial && make)
echo "Making parallel"
(cd parallel && make)
echo
echo "Running serial"
time (./serial/PhonePricePrediction.out dataset/)
echo
echo "Running parallel"
time (./parallel/PhonePricePrediction.out dataset/)
\end{lstlisting}
\RTL

بعد از اجرای این اسکریپت خروجی زیر دریافت شده که نشان می‌دهد زمان اجرای برنامه به مراتب سریع‌تر شده.

\LTR
\begin{lstlisting}[language=bash]
Running serial
Accuracy: 84.65%
real    0m0.022s
user    0m0.015s
sys     0m0.007s

Running parallel
Accuracy: 83.4%
real    0m0.012s
user    0m0.016s
sys     0m0.003s
\end{lstlisting}
\RTL
\end{document}
